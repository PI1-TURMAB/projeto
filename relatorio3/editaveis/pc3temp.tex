\section{PC03 provisório}

\subsection{Objetivo}

O foco do grupo de estruturas é elaborar e calcular toda a parte mecânica do projeto, bem como definir o sistema de bombeamento e arrefecimento. É de extrema importância para o projeto e para todos os subgrupos que, seja efetuada de maneira precisa a montagem da estrutura e da sua análise de risco e orçamento.

\subsection{Metodologia}

O grupo levou em conta 7 (sete) aspectos referentes à parte estrutural e, a partir daí, elaborou as pesquisas necessárias para decidir o melhor modelo para o sistema de arrefecimento. Os aspectos são:

\begin{itemize}
\item	Líquido: Nessa parte, o grupo realizou o estudo das opções de líquido disponíveis para arrefecimento do sensor. Foi feito o estudo em conjunto com o grupo de Transmissão de Calor.
\item	Resfriador: Nessa parte foram avaliadas as opções de radiadores, serpentinas e outros métodos de perda de calor. Foi feito o estudo em conjunto com o grupo de Transmissão de Calor.
\item	Bomba: Foram estudadas diferentes bombas para o sistema. Nesta fase serão levados em consideração a potência da bomba, a vazão e a pressão nas quais ela trabalha.
\item Reservatório: Esta etapa consistiu em decidir o tamanho do reservatório necessário para o funcionamento do teste e para o resfriamento pós teste (total de 2 min aproximadamente)
\item	Alimentação: Foi a parte de decisão do sistema elétrico e eletrônico e, sobretudo, avaliar a demanda energética do projeto. Esta fase será realizada em conjunto ao grupo de Processamento de Dados.
\item	Condutores: Esta etapa consistiu em avaliar o material e as dimensões dos tubos e conectores entre o sistema de arrefecimento e o sensor.
\item	Bancada: Obrigatoriamente a última etapa, consistiu em avaliar a organização do protótipo junto à bancada de testes nos quesitos ergonomia e segurança.
\end{itemize}

Durante as pesquisas realizadas pelo grupo, constaram inúmeras hipóteses de montagem do sistema que estarão descritas ao longo do relatório. Com o objetivo de facilitar e agilizar as pesquisas realizadas, o grupo se subdividiu em 5 (cinco) subgrupos, sendo estes: projeto conceitual de bombeamento e arrefecimento, projeto conceitual de tubulações e conectores, projeto conceitual do líquido de arrefecimento, avaliação de riscos técnicos nos sistemas de bombeamento e arrefecimento e avaliação de riscos técnicos nas tubulações e conectores. 

Além disso, foi desenvolvido um modelo em 3D do sistema na plataforma CATIA.

\subsection{Introdução}

A montagem geral da estrutura do projeto será baseada no esquemático da Figura 1. Nele constam: 4 (quatro) coolers, um reservatório, duas válvulas de redirecionamento de fluxo, duas bombas e a tubulação (incluindo os conectores não ilustrados). O projeto consiste em bombear a água do reservatório para o sensor da Kistler, que aquecerá o líquido. Após o contato com o sensor, a água retorna ao sistema passando pelo radiador, onde será resfriada e adicionada de volta ao reservatório.

\subsection{Desenvolvimento}

\subsubsection{Sistema de bombeamento e arrefecimento}

Como base para o projeto conceitual do sistema de arrefecimento e bombeamento utilizamos, conforme sugerido pelo professor (Anexo 1), o uso de um sistema de water cooling semelhante ao utilizado no resfriamento de processador para computadores. Partindo dessa premissa, foram avaliadas as seguintes possibilidades:

\begin{itemize}
\item Utilização de um watercooler comercial.
\item Montagem de um watercooler com peças avulsas.
\end{itemize}

Sobre o primeiro tópico, o uso de um watercooler comercial foi avaliado tendo em vista três modelos: um considerado de acesso, um intermediário e um top de linha. Já sobre o segundo tópico, tivemos em mente a obtenção do melhor custo-benefício, com objetivo de comparar com os watercoolers comerciais estudados previamente e com os dados obtidos do sistema de resfriamento proprietário da Kistler.

\subsubsection{Líquido de arrefecimento}

O líquido de arrefecimento tem como função principal trocar calor com o sensor, refrigerando o sistema para evitar o sobreaquecimento.

Para isso o fluido de arrefecimento percorre a tubulação do sistema sem entrar em contato direto com os componentes eletrônicos do sensor, assim ele é resfriado graças à troca de calor entre a água (temperatura ambiente) e o sensor estimado em no máximo $450^{\circ}C$.  Para que haja um fluxo controlado de água, será usado um sistema de bombeamento, garantindo a troca efetiva de calor.

O líquido de arrefecimento será a água pois, além de ser um fluido de facil obtenção, possui propriedades térmicas e viscosas compatíveis com os requisitos do projeto a ser desenvolvido.

Sua capacidade térmica é bem elevada quando comparada com outras substâncias conhecidas (aproximadamente $1cal/g^{\circ}C$), além de possuir uma viscosidade baixa como é possível observar na tabela retirada de [1].

A água é um liquido em abundância e que pode ser facilmente encontrado. O litro de aguá mineral, que possui poucas impurezas, é cerca de um real e, a água destilada pode ser encontrada a R\$1,95 o litro [3]. Portanto, é um líquido de arrefecimento com pontos econômicos muito positivos.

\subsubsection{Tubulação}
