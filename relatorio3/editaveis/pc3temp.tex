\section{PC03 provisório}

\subsection{Objetivo}

O foco do grupo de estruturas é elaborar e calcular toda a parte mecânica do projeto, bem como definir o sistema de bombeamento e arrefecimento. É de extrema importância para o projeto e para todos os subgrupos que, seja efetuada de maneira precisa a montagem da estrutura e da sua análise de risco e orçamento.

\subsection{Metodologia}

O grupo levou em conta 7 (sete) aspectos referentes à parte estrutural e, a partir daí, elaborou as pesquisas necessárias para decidir o melhor modelo para o sistema de arrefecimento. Os aspectos são:

\begin{itemize}
\item	Líquido: Nessa parte, o grupo realizou o estudo das opções de líquido disponíveis para arrefecimento do sensor. Foi feito o estudo em conjunto com o grupo de Transmissão de Calor.
\item	Resfriador: Nessa parte foram avaliadas as opções de radiadores, serpentinas e outros métodos de perda de calor. Foi feito o estudo em conjunto com o grupo de Transmissão de Calor.
\item	Bomba: Foram estudadas diferentes bombas para o sistema. Nesta fase serão levados em consideração a potência da bomba, a vazão e a pressão nas quais ela trabalha.
\item Reservatório: Esta etapa consistiu em decidir o tamanho do reservatório necessário para o funcionamento do teste e para o resfriamento pós teste (total de 2 min aproximadamente)
\item	Alimentação: Foi a parte de decisão do sistema elétrico e eletrônico e, sobretudo, avaliar a demanda energética do projeto. Esta fase será realizada em conjunto ao grupo de Processamento de Dados.
\item	Condutores: Esta etapa consistiu em avaliar o material e as dimensões dos tubos e conectores entre o sistema de arrefecimento e o sensor.
\item	Bancada: Obrigatoriamente a última etapa, consistiu em avaliar a organização do protótipo junto à bancada de testes nos quesitos ergonomia e segurança.
\end{itemize}

Durante as pesquisas realizadas pelo grupo, constaram inúmeras hipóteses de montagem do sistema que estarão descritas ao longo do relatório. Com o objetivo de facilitar e agilizar as pesquisas realizadas, o grupo se subdividiu em 5 (cinco) subgrupos, sendo estes: projeto conceitual de bombeamento e arrefecimento, projeto conceitual de tubulações e conectores, projeto conceitual do líquido de arrefecimento, avaliação de riscos técnicos nos sistemas de bombeamento e arrefecimento e avaliação de riscos técnicos nas tubulações e conectores. 

Além disso, foi desenvolvido um modelo em 3D do sistema na plataforma CATIA.

\subsection{Introdução}

A montagem geral da estrutura do projeto será baseada no esquemático da Figura 1. Nele constam: 4 (quatro) coolers, um reservatório, duas válvulas de redirecionamento de fluxo, duas bombas e a tubulação (incluindo os conectores não ilustrados). O projeto consiste em bombear a água do reservatório para o sensor da Kistler, que aquecerá o líquido. Após o contato com o sensor, a água retorna ao sistema passando pelo radiador, onde será resfriada e adicionada de volta ao reservatório.

\subsection{Desenvolvimento}

\subsubsection{Sistema de bombeamento e arrefecimento}

Como base para o projeto conceitual do sistema de arrefecimento e bombeamento utilizamos, conforme sugerido pelo professor (Anexo 1), o uso de um sistema de water cooling semelhante ao utilizado no resfriamento de processador para computadores. Partindo dessa premissa, foram avaliadas as seguintes possibilidades:

\begin{itemize}
\item Utilização de um watercooler comercial.
\item Montagem de um watercooler com peças avulsas.
\end{itemize}

Sobre o primeiro tópico, o uso de um watercooler comercial foi avaliado tendo em vista três modelos: um considerado de acesso, um intermediário e um top de linha. Já sobre o segundo tópico, tivemos em mente a obtenção do melhor custo-benefício, com objetivo de comparar com os watercoolers comerciais estudados previamente e com os dados obtidos do sistema de resfriamento proprietário da Kistler.

\subsubsection{Líquido de arrefecimento}

O líquido de arrefecimento tem como função principal trocar calor com o sensor, refrigerando o sistema para evitar o sobreaquecimento.

Para isso o fluido de arrefecimento percorre a tubulação do sistema sem entrar em contato direto com os componentes eletrônicos do sensor, assim ele é resfriado graças à troca de calor entre a água (temperatura ambiente) e o sensor estimado em no máximo $450^{\circ}C$.  Para que haja um fluxo controlado de água, será usado um sistema de bombeamento, garantindo a troca efetiva de calor.

O líquido de arrefecimento será a água pois, além de ser um fluido de facil obtenção, possui propriedades térmicas e viscosas compatíveis com os requisitos do projeto a ser desenvolvido.

Sua capacidade térmica é bem elevada quando comparada com outras substâncias conhecidas (aproximadamente $1cal/g^{\circ}C$), além de possuir uma viscosidade baixa como é possível observar na tabela retirada de [1].

A água é um liquido em abundância e que pode ser facilmente encontrado. O litro de aguá mineral, que possui poucas impurezas, é cerca de um real e, a água destilada pode ser encontrada a R\$1,95 o litro [3]. Portanto, é um líquido de arrefecimento com pontos econômicos muito positivos.

\subsubsection{Tubulação}
\paragraph{Tamanho}
É necessário um correto dimensionamento da sala onde se encontra o equipamento, ou seja:
\begin{itemize}
\item Posição do sensor na bancada de teste (Conhecido).
\item Posição do sistema de arrefecimento (Conhecido – 3 Opções).
\end{itemize}

\paragraph{Material}

O material escolhido para a tubulação é o politetrafluoretileno (PTFE), família dos fluoroplásticos (PTFE, FEP, PFA, CTFE, ECTFE, ETFE, PVDF), que são resinas termoplásticas de estrutura principal parafínica que têm alguns ou todos os seus hidrogênios substituídos por átomos de flúor $(C_{2}F_{4})_{n}$. Caracteriza-se por excelentes propriedades dielétricas e resistência química, baixo coeficiente de atrito e excepcional estabilidade em elevadas temperaturas, resistência mecânica baixa ou moderada, custo elevado.

O material foi escolhido por possuir excelentes propriedades, que são indispensáveis para o transporte do fluido de arrefecimento, bem como por contar com boa margem de segurança, já que todas as propriedades do material são superiores às necessitadas no projeto.

\begin{itemize}
\item 	Temperatura de trabalho: possuem uma das mais amplas faixas de temperatura de trabalho, desde $-90^{\circ}C$ até $+240^{\circ}C$, em regime contínuo. A melhor opção em mangueiras para trabalho com vapor superaquecido.
\item Deslizamento sem atrito: com baixo coeficiente de atrito, são particularmente indicados em sistemas mecânicos que apresentam dificuldade de lubrificação periódica.
\item Resistência à água e intempéries: são altamente hidrorrepelentes, apresentando absorção de umidade praticamente nula. Resiste aos raios solares, variações de umidade relativa do ar e da temperatura ambiente. 
\item Baixa aderência: por ser antiaderente, o PTFE é o polímero adequado para tubos e mangueiras que conduzem líquidos viscosos e adesivos. 
\end{itemize}

Além disso, o material possui excelente custo benefício, visto que o comprimento da tubulação é relativamente pequeno. Sendo o comprimento máximo (mais detalhado a frente no relatório) é de 7,8 metro e o custo por metro é de R\$4,50.

\paragraph{Conectores}

Os conectores são importantíssimos para a correta montagem da tubulação, por serem conhecidos como os elos fracos na montagem da mesma, o grupo analisou e concordou com a utilização de conectores com melhor qualidade, mesmo tendo um custo mais alto, que se justifica pela vida útil e margem de segurança que os mesmos trazem ao sistema.

Os conectores escolhidos são da marca Swagelok, uma multinacional americana com mais de 50 anos de produção em conectores. Esses conectores possuem um tipo de travagem que fornece:
\begin{itemize}
\item Excelente vedação para gás e ação de crimpagem no tubo.
\item Fácil realização de corretas instalações.
\item Reapertos consistentes.
\item Excelente resistência à fadiga por vibrações e suporte do tubo, etc.
\end{itemize}

Como colocado acima a utilização dos mesmos é explicada por terem melhor resistência a:

\begin{itemize}
\item Vazamentos
\item Vibração (cravamento no tubo).
\item Choque térmico
\item Corrosão
\end{itemize}

A Swagelok produz os conectores a partir dos mais diversos materiais, referente contato com a parte técnica da empresa foi conseguido documentos técnicos sobre valores máximos para temperatura e pressão, bem como o custo dos conectores de cada tipo de material, com isso pode-se realizar a correta escolha dos conectores para a utilização.

O material escolhido é o latão, seguindo o conselho do responsável técnico da Swagelok.

Com todos os dados em mãos é possível determinar margem de segurança em relação a pressão (99\%) e temperatura (50\%).

Como informado anteriormente no relatório, temos três opções para implementação do sistema, que variam a quantidade de conectores. Em integração com o grupo de controle, discutimos quais conectores seriam precisos para a integração dos sistemas, após a discussão chegou-se ao consenso de quais conectores utilizar, independente da opção de implementação escolhida, o sistema precisa dos conectores listados abaixo:

\begin{itemize}
\item 5 Redutores (2-Sensor de Fluxo, 2-Reservatório e 1-Sensor de Temperatura).
\item 2 T (Sensor de Pressão e de Temperatura).
\item 2 Machos (Radiador).
\end{itemize}

A opção de implementação traz a variação da quantidade de cotovelos $90^{\circ}$, tal variação é especificada na tabela abaixo.

\begin{table}[htb]
\centering
\begin{tabular}{|p{2cm}|p{2cm}|}
\hline
Opção & Quantidade de cotovelos \\
\hline
1 & 4 \\ \hline
2 & 2 \\ \hline
3 & 2 \\ \hline
\hline
\end{tabular}
\end{table}

\paragraph{Reservatório}

Seguindo a dica do professor sobre a utilização de materiais COTS, o reservatório escolhido é um reservatório de radiador automotivo, que se enquadra nas especificações necessárias, por ser fabricado em material de qualidade, com boa resistência a temperatura, além de contar com boa disponibilidade e baixo custo.

A principal informação a se ter sobre o reservatório é sua capacidade, para o em questão a mesma é na faixa de 2 litros, tal capacidade é mais que necessária para o projeto, já que a partir de cálculos simples podemos determinar a quantidade de fluido em toda tubulação quando o sistema está hermeticamente fechado, uma condição primordial para a operação do sistema. Como informado anteriormente no relatório, temos três opções para implementação do sistema, que variam o comprimento da mangueira, modificando assim o volume de água no sistema.

\begin{table}[htb]
\centering
\begin{tabular}{|p{2cm}|p{2cm}|p{3cm}|}
\hline
Opção & Comprimento da mangueira (m) & Quantidade de fluxo na tubulação (ml) \\
\hline
1 & 6 & 42,40 \\ \hline
2 & 6,5 & 45,93 \\ \hline
3 & 7,8 & 54,40 \\ \hline
\hline
\end{tabular}
\end{table}

Com a ajuda da tabela acima podemos observar que a quantidade máxima de fluido presa na tubulação é 54,4 ml. O reservatório conta com a capacidade na faixa de 2 litros dando ao sistema uma margem de segurança na casa dos 90\%. 

O reservatório em questão é produzido a partir do PEAD (Polietileno de Alta Densidade) que conta densidade igual ou maior que 0,941 $g/(cm)^3$, um baixo nível de ramificações, com alta densidade e altas forças intermoleculares, trazendo características como:

\begin{itemize}
\item Resistente a altas temperaturas.
\item Alta resistência à tensão, compressão e tração;
\item Baixa densidade em comparação com metais e outros materiais.
\item Impermeável.
\item Inerte (ao conteúdo), baixa reatividade.
\item Atóxico.
\item Pouca estabilidade dimensional.
\end{itemize}

O reservatório conta com dimensões gerais: 15 x 33 x 22 cm, e peso total (vazio): 355 gramas.

\paragraph{Desenvolvimento do modelo em 3D}

Para o desenvolvimento do modelo em 3D do sistema de arrefecimento foi utilizada a plataforma CATIA e, nesse modelo, foi priorizado o destaque à organização do sistema como um todo.
