\chapter{Conclusão}
 Este documento visou apresentar uma definição técnica mais profunda do projeto a ser desenvolvido pela turma B da disciplina de Projeto Integrador para Engenharia 1, do campus Gama da Universidade de Brasília. Neste PC3 (Ponto de Controle 3) foram retratados: aspectos econômicos finais da implementação, simulações de testes para o projeto como um todo, análise para a criação de simulações mais próximas do sistema em funcionamento, esquemático (em CATIA) do projeto integral, melhorias necessárias -sendo estas orçamentais e técnicas- além do protótipo do produto final. Após o PC2 (Ponto de Controle 2), todas as tarefas e objetivos estipulados para o PC3 foram cumpridas, como também houve melhorias no quesito de comunicação do grupo como um todo e eficiência dos subgrupos. Tendo em vista o produto final, o projeto conta com o desenvolvimento de um protótipo, posto que possivelmente possa ser aplicado para outros modelos que utilizem o mesmo sensor de pressão Kistler 6061BS32.  
