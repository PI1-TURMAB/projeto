\chapter{Introdução}
O presente relatório é elaborado no âmbito da disciplina Projeto Integrador de Engenharias I, com vista a primeira oportunidade de entrar em contato com a complexidade de um projeto real de engenharia no mercado de trabalho, de forma a complementar e aperfeiçoar as competências até então adquiridas pelos alunos presentes, através de uma ligação entre o sistema educativo e o mundo laboral. O tema deste projeto se baseia no desenvolvimento do projeto de implementação de um sistema de resfriamento para o sensor de pressão Kistler 6061BS32 (e variações), que futuramente será utilizado no motor de foguete híbrido do Laboratório de Propulsão Híbrida da Faculdade do Gama (FGA).É importante ressaltar a importância do desenvolvimento deste projeto, tendo em vista que é fundamental para o funcionamento efetivo das medições do motor de combusão interna, devido à alta temperatura (superior à 1000 Kelvin), assim como a disponibilidade de um sistema de arrefecimento para este tipo de sensor ser, de certa forma, baixa e com preços elevados (um dos disponíveis alcança a marca dos R\$30.000. Até o próximo Ponto de Controle (PC3), serão feitas análises de orçamento e mercado para, possivelmente, disponibilizar o sistema desenvolvido para o mercado, como uma alternativa econômica, tanto para motores de combusão interna como para sensores que necessitam de exposição a altas temperaturas. Para efeitos de avanço do projeto para o PC2 (Ponto de Controle 2), serão apresentados tópicos desde a organização da equipe até as soluções propostas com suas especificações técnicas, além das análises de risco para cada área.

