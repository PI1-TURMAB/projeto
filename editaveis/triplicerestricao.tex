\chapter{Tríplice Restrição no âmbito do Projeto}

\section{Escopo}

A premissa do projeto é criar um sistema de arrefecimento para o sensor de pressão kistler modelo 6061B, que se adapte às condições do motor de foguete híbrido do laboratório de propulsão da Unb Gama. A partir disso foi definido o escopo do projeto da seguinte forma:

\subsection{Objetivos que Fazem parte do escopo proposto:}

\begin{itemize}

	\item A definição dos sensores e microcontroladores a serem utilizados no sistema de acordo com os parâmetros de funcionamento do motor da FGA;

	\item Desenvolver simulações;

	\item Programar o sistema de controle;

	\item Implementar um protótipo;

	\item Definir a estrutura do sistema e o líquido de resfriamento a ser utilizado;

	\item Avaliar a viabilidade de construção do resfriador e entregar a documentação de pesquisa e embasamento teórico gerada durante o projeto.

\end{itemize} 

\subsection{Objetivos que não Fazem parte do escopo proposto:}

\begin{itemize}

	\item Construir o resfriador final.

\end{itemize}

\section{Tempo}

Iniciou-se o projeto no dia 30 de agosto de 2017 com a definição do grupo de trabalho e a data final para entrega é o dia 4 de dezembro de 2017. Considerando-se esse intervalo de tempo foi construído um cronograma de atividades a serem executadas para o sucesso do projeto.
O cronograma está dividido em atividades determinadas para cada ponto de controle, contendo desde tarefas que envolvem todo grupo, a tarefas específicas de cada área, foi confeccionado com a ferramenta Gantter e está ilustrado na figura abaixo, contendo as datas e uma visualização fornecida pela ferramenta.

\begin{figure}[!htb]
   \centering
   \includegraphics[width=15cm, keepaspectratio=true]{figuras/cronogramageral.eps}
   \caption{Visão geral do cronograma}
\end{figure}


\begin{figure}[!htb]
   \centering
   \includegraphics[width=15cm, keepaspectratio=true]{figuras/relacaotarefas.eps}
   \caption{Visão de dependência entre as tarefas e prazos}
\end{figure}
