\chapter{Requisitos do Projeto}

\section{Requisitos gerais do Projeto}

\begin{itemize}
\item Formatação dos documentos: A elaboração e manutenção dos documentos produzidos no projeto deverá utilizar LaTeX de forma que a apresentação das informações fique organizada.
\item Simulações: O projeto deve possuir simulações a fim de garantir o funcionamento correto do protótipo e facilitar possíveis mudanças.
\item Custo: O projeto deve ser viável economicamente para o escopo da disciplina e restrições da universidade.
\item Funcionalidade: O sistema não pode interferir nos sinais dos sensores instalados no motor para evitar possíveis acidentes e erros de funcionamento.
\item Protótipo: O protótipo resultante do projeto deve ser robusto e flexível.
\end{itemize}

\subsection{Requisitos da equipe de estruturas}
\begin{itemize}
\item Materiais: Os materiais a serem utilizados no Projeto deverão preferencialmente ser adquiridos na região limítrofe do Distrito Federal e considerando o menor custo desde que atenda às necessidades exigidas.
\item Estruturas: O sistema deve possuir um reservatório para o líquido do sistema de arrefecimento e uma bomba para a circulação do mesmo.
\end{itemize}

\subsection{Requisitos da equipe de controle}
\begin{itemize}
\item Arrefecimento: O sistema de arrefecimento deve possuir  um sistema de controle de vazão do liquido de arrefecimento além de um projeto elétrico para o controle da bomba de líquido.
\item Estrutura: a estrutura em torno e próximo ao sensor principal(sensor de pressão Kistler 6061B), visando encontrar a melhor forma e estrutura para a instalação de um sensor de temperatura na parte externa próxima ao sensor.
\item Eficiência: potência do resfriador utlizado, visando ter um melhor controle da temperatura do sistema. 
\end{itemize}

\subsection{Requisitos da equipe de processamento de dados}
\begin{itemize}
\item Simulação: As simulações realizadas no projeto deverão ser feitas a partir do software LabVIEW para a obtenção de resultados confiáveis e mais próximos à realidade.
\item Atualização: O sistema deve realizar a obtenção e o compartilhamento de dados em tempo real para possibilitar controle imediato.
\end{itemize}

\subsection{Requisitos da equipe de transmissão de calor}
\begin{itemize}
\item Temperatura: para efeitos de funcionamento efetivo do sensor, a temperatura não pode se elevar até mais que 250ºC, assim levando em consideração uma margem de erro e a sensibilidade do sensor. 
\item Líquido: água ou fluído com propriedades semelhantes.
\item Funcionalidade: 2 pontos da câmara de combustão, levando em consideração uma distância pequena suficiente para não interferir na medição e grande o suficiente para demorar a propagar o calor até o sensor
\item Materiais: O sistema de resfriamento deverá utilizar materiais COTS(Commercial off-the-shelf).
\end{itemize}

\subsection{Requisitos da equipe de gerência}
\begin{itemize}
\item Software: O controle de repositórios e arquivos do projeto deverá feito através da plataforma GitHub a fim de facilitar e armazenar os produtos e documentos do projeto. As atividades devem ser organizadas utilizando Kanban por meio da extensão ZenHub. A elaboração dos cronogramas deve ser feito no Gantter para uma apresentação concisa dos dados.
\end{itemize}
