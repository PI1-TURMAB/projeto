\chapter{Metodologia e organização da equipe}

A primeira atividade após a formação dos grupos foi a definição das áreas de trabalho no projeto de forma que fosse possível bolar uma metodologia a ser seguida para atingir um aproveitamento maior no projeto.
Após duas reuniões a equipe optou pela separação do grupo que contém em média trinta e cinco pessoas em cinco áreas de trabalho, sendo elas:

\begin{itemize}
\item Controle: Execução do projeto do sistema de controle de temperatura, sistema de arrefecimento
\item Transmissão de Calor: A premissa do projeto é criar um sistema de arrefecimento que se adapte às situações termodinâmicas encontradas na câmara de combustão de um sistema híbrido de um simulador.
\item Processamento de dados: Tem como objetivo elaborar uma estratégia para controlar o sistema de arrefecimento a partir dos dados fornecidos pelo grupo de controle.
\item Estrutura: O objetivo do grupo de estrutura é entregar um modelo teórico e prático do funcionamento mecânico do sistema de arrefecimento.
\item Gerência de projeto: Tem como objetivo alinhar as equipes, organizar atividades e tratar de artefatos como requisitos, custos do projeto.
\end{itemize}

Com o intuito de organizar tais equipes, foram definidos um gerente geral, e os gerentes de cada área de trabalho, onde a cada semana será feita reuniões para alinhamento do projeto, nas aulas de quarta-feira, onde os líderes se encontram e repassam o que cada grupo fez na semana anterior, oferecendo propostas de melhorias e levantamento de possíveis problemas que foram encontrados.
A idéia é cada grupo seguir a metodologia ágil, baseando-se no SCRUM e Kanban, a fim de que o gerente de cada equipe determine as tarefas, e possa acompanhar o andamento dessas atividades através de reuniões diárias com sua equipe.
A equipe de gerência se dividirá para que cada integrante da mesma possa acompanhar os grupos de forma individual para possíveis levantamentos de soluções em dificuldades encontradas.