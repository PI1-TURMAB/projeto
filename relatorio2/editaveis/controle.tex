\chapter{Equipe de Controle}
\section{Introdução}
Considerando o escopo do projeto, o grupo responsável pela parte de controle, associado às informações elaboradas pelos demais grupos apresentará, por meio deste relatório, uma proposta de construção do sistema a partir da utilização de sensores para o monitoramento dos parâmetros: temperatura, pressão, fluxo e nível.\\
Observando as duas propostas apresentadas anteriormente, onde um sensor de temperatura principal seria auxiliado por sensores de temperatura de redundância, tem-se que a proposta definida é semelhante. A diferença está no foco dado à cada sensor em diferentes períodos do teste do foguete híbrido. Durante a realização do teste, os sensores de temperatura internos (localizados no reservatório e na tubulação) serão analisados com maior atenção, a fim de manter a faixa de temperatura no nível de bom funcionamento do sensor de pressão da Kistler. Após o encerramento do teste, o foco principal estará no sensor de temperatura externo (localizado no adaptador da câmara de combustão), já que a transferência de calor que se dá após o teste é alta e deve-se controlar a temperatura do sistema para que os componentes mantenham sua integridade.\\
Diante disso, há no mercado uma gama enorme de sensores que podem ser escolhidos para o funcionamento de um projeto, e a escolha destes deve levar em consideração não só questões operacionais do produto requisitado, mas também uma relação custo-benefício. Além do mais, criticidade do projeto demanda o uso de sub-sistemas redundantes, garantindo maior confiabilidade ao controle do sistema.\\
Portanto, considerando os requisitos necessários para o bom funcionamento do sistema, foram definidos seis sensores, sendo eles três de temperatura, um de pressão, um de fluxo e um de nível, que realizarão o monitoramento dos parâmetros cruciais. Serão utilizados também um microcontrolador, bem como circuitos paralelos de comunicação, amplificação de sinal e alimentação.\\
