\chapter{Equipe de Controle}
\section{Introdução}
Considerando o escopo do projeto, o grupo responsável pela parte de controle, associado às informações elaboradas pelos demais grupos apresentará, por meio deste relatório, uma proposta de construção do sistema a partir da utilização de sensores para o monitoramento dos parâmetros: temperatura, pressão, fluxo e nível.\\
Observando as duas propostas apresentadas anteriormente, onde um sensor de temperatura principal seria auxiliado por sensores de temperatura de redundância, tem-se que a proposta definida é semelhante. A diferença está no foco dado à cada sensor em diferentes períodos do teste do foguete híbrido. Durante a realização do teste, os sensores de temperatura internos (localizados no reservatório e na tubulação) serão analisados com maior atenção, a fim de manter a faixa de temperatura no nível de bom funcionamento do sensor de pressão da Kistler. Após o encerramento do teste, o foco principal estará no sensor de temperatura externo (localizado no adaptador da câmara de combustão), já que a transferência de calor que se dá após o teste é alta e deve-se controlar a temperatura do sistema para que os componentes mantenham sua integridade.\\
Diante disso, há no mercado uma gama enorme de sensores que podem ser escolhidos para o funcionamento de um projeto, e a escolha destes deve levar em consideração não só questões operacionais do produto requisitado, mas também uma relação custo-benefício. Além do mais, criticidade do projeto demanda o uso de sub-sistemas redundantes, garantindo maior confiabilidade ao controle do sistema.\\
Portanto, considerando os requisitos necessários para o bom funcionamento do sistema, foram definidos seis sensores, sendo eles três de temperatura, um de pressão, um de fluxo e um de nível, que realizarão o monitoramento dos parâmetros cruciais. Serão utilizados também um microcontrolador, bem como circuitos paralelos de comunicação, amplificação de sinal e alimentação.\\
\section{Desenvolvimento}
O sistema de controle foi baseado em cinco partes, sendo elas: coleta de dados, amplificação de sinais, análise de dados, transferência de dados e controle de resposta.
\subsection{Coleta de Dados}
A parte de coleta de dados será realizada pelos sensores e será a porta de entrada para os dados de monitoramento dos principais parâmetros durante a realização do teste. O sensor de pressão da Kistler estará em funcionamento juntamente com todo o sistema de arrefecimento durante o teste. Sendo assim os principais parâmetros a serem monitorados são temperatura (sensor de pressão da Kistler, tubulação e reservatório), pressão (tubulação), fluxo (tubulação) e nível (reservatório). Portanto, após diversas pesquisas, os modelos mais adequados para cumprirem essas funções foram:
\begin{itemize}
	\item {\bf Sensor de Temperatura para o adaptador da câmara de combustão}:
	\begin{itemize}
		\item {\bf Modelo:} TJ36-CASS-116G-6-CC-XCIB;
		\item{\bf Descrição:} Sonda TJ36 com cabo de ligação com isolamento cerâmico de Nextel® e revestimento trançado de Inconel® 600, terminações desencapadas;
		\item{\bf Material da Bainha:} sonda em aço inox 304;
		\item{\bf Categoria do termopar:} tipo K;
		\item{\bf Tipo de Junção:} Aterrada;
		\item{\bf Dimensões:}
		\begin{itemize}
			\item  Diâmetro da sonda de 1,59 mm (1/16” pol);
			\item Comprimento de 150 mm (6” pol);
			\item Comprimento do fio de 1,0 m.
		\end{itemize}
		\item {\bf Temperatura de operação:} -200 ºC a 899 ºC;
		\item {\bf Preço:} R\$382,00.
	\end{itemize}
	
\end{itemize}
