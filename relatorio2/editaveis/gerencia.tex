\section{Gerência}

O foco do grupo de gerência foi além de integrar os demais grupos, elaborar e assegurar a utilização dos modelos de análise e respostas aos riscos.

\subsection{Modelo de documentação de riscos}

\subsubsection{Registro dos riscos}
Abaixo está o modelo que utilizamos para \textbf{registrar} os riscos encontrados no nosso projeto. E tambem esta descrito abaixo as tabelas auxiliares para o preenchimento das colunas de \textbf{impacto}, \textbf{probabilidade}.

\begin{table}[h]
    \centering
    \begin{tabular}{|p{1cm}|p{2cm}|p{2cm}|p{2cm}|p{2cm}|p{1.7cm}|p{2.9cm}|}
    \hline
    \textbf{ID}  & \textbf{Risco} & \textbf{Tipo de Risco} & \textbf{Descrição} & \textbf{Descrição do Impacto} & \textbf{Impacto} & \textbf{Probabilidade} \\ \hline
\end{tabular}
    \caption{Modelo para registro dos riscos}
    \end{table}

    \begin{table}[h]
        \centering
        \begin{tabular}{|p{3cm}|p{3cm}|}
        \hline
        \textbf{Probabilidade}  & \textbf{Intervalo} \\ \hline
        Muito baixa & Menor que 20\% \\ \hline
        Baixa & 21\% a 40\% \\ \hline
        Média & 41\% a 60\% \\ \hline
        Alta & 61\% a 80\%\\ \hline
        Muito alta & Acima de 80\%\\ \hline
    \end{tabular}
        \caption{Tabela auxiliar para preenchimento da probabilidade dos riscos}
        \end{table}

        \begin{table}[h]
            \centering
            \begin{tabular}{|p{3cm}|p{3cm}|}
            \hline
            \textbf{Impacto}  & \textbf{Descrição} \\ \hline
            Muito baixo & Não afeta significativamente o projeto \\ \hline
            Baixo & Afeta o projeto mas é facilmente contornado \\ \hline
            Médio & Pode causar atrasos ao projeto mas não são tão significativos \\ \hline
            Alto & Pode causar significativos atrasos no projeto \\ \hline
            Muito alto & Pode impedir o projeto de ser concluído \\ \hline
        \end{tabular}
            \caption{Tabela auxiliar para preenchimento do impacto dos riscos}
            \end{table}

\subsubsection{Respostas aos riscos}
Abaixo está o modelo que utilizamos para \textbf{respostas} aos riscos encontrados no nosso projeto. E tambem esta descrito abaixo as tabelas auxiliares para o preenchimento das colunas de \textbf{prioridade}.

\begin{table}[h]
    \centering
    \begin{tabular}{|p{3cm}|p{3cm}|p{3cm}|p{3cm}|}
    \hline
    \textbf{ID}  & \textbf{Descrição} & \textbf{Ação} & \textbf{Prioridade} \\ \hline
\end{tabular}
    \caption{Modelo de documentação para as repostas aos riscos}
    \end{table}

\begin{table}[h]
    \centering
    \begin{tabular}{|p{3cm}|p{3cm}|}
    \hline
    \textbf{Escala}  & \textbf{Intervalo} \\ \hline
    Baixa & 1 a 5 \\ \hline
    Média & 6 a 14 \\ \hline
    Alta & 15 a 25 \\ \hline
\end{tabular}
    \caption{Tabela auxiliar para preenchimento da prioridade das respostas aos riscos}
    \end{table}

            