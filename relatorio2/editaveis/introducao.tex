\chapter{Introdução}
O presente relatório é elaborado no âmbito da disciplina Projeto Integrador de Engenharias I, com vista a primeira oportunidade de entrar em contato com a complexidade de um projeto real de engenharia no mercado de trabalho, de forma a complementar e aperfeiçoar as competências até então adquiridas pelos alunos presentes, através de uma ligação entre o sistema educativo e o mundo laboral. O tema deste projeto se baseia no desenvolvimento do projeto de implementação de um sistema de resfriamento para o sensor de pressão Kistler 6061BS32 (e variações), que futuramente será utilizado no motor de foguete híbrido do Laboratório de Propulsão Híbrida da Faculdade do Gama (FGA). Para efeitos de avanço do projeto para o PC2 (Ponto de Controle 2), serão apresentados tópicos desde a organização da equipe até as soluções propostas com suas especificações técnicas.

