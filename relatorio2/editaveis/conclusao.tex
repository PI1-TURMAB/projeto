\chapter{Conclusão}
  Este documento visou apresentar uma definição técnica mais profunda do projeto a ser desenvolvido pela turma B da disciplina de Projeto Integrador para Engenharia 1, do campus Gama da Universidade de Brasília. Além das propostas de organização, requisitos elicitados, cronograma a ser seguido e possíveis soluções para o problema a ser resolvido já apresentadas no PC1(Ponto de Controle 1), também foram denotadas as especificações técnicas de todos os materiais necessários para a implementação do produto. Após o PC1, todas as tarefas e objetivos estipuladas para o PC2 foram cumpridas, como também houveram melhorias no quesito de comunicação do grupo como um todo e eficiência dos subgrupos. Tendo em vista o produto final, o projeto ainda precisa de melhorias, sendo elas, principalmente, orçamentais e técnicas, posto que o desenvolvimento deste produto, possivelmente, pode ser aplicado para outros modelos que utilizem o mesmo sensor de pressão Kistler 6061BS32. Assim, fazem-se necessárias mudanças na organização dos grupos internos, sendo possível a criação de novos e a supressão de outros, tendo em vista a mudança que o grupo realizou no andar do projeto. Para o próximo relatório (Ponto de Controle 3), serão retradados os aspectos econômicos finais da implementação, além das simulações de testes para o projeto como um todo e a análise para a criação de simulações mais próximas do sistema em funcionamento, além do esquemático (em CATIA) do projeto integral.   
