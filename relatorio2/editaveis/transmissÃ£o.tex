\section{Transmissão de Calor}

O objetivo do grupo de transmissão de calor, é auto explicativo. Determinar as temperaturas e influências térmicas existentes no sistema, prioritariamente no que tange à esfera da câmara de combustão e suas derivadas. Assim como analisar os riscos, determinar diretrizes à outros grupos e custos de materiais.

\subsection{Metodologia de trabalho}

Seguindo o seu objetivo, primeiro foram determinados o que seria de importância ao desenvolvimento do projeto como um todo, posteriormente o grupo de subdividiu em unidades individuais de forma que mais pesquisa pudesse ser realizada em menos tempo. Posteriormente foi subdividido novamente em tais equipes, conceito e documentação, simulações, tubulações e modelos matemáticos e físicos aplicáveis.\\
A determinação precisa do fluxo de calor é uma importante tarefa tanto para o projeto quanto para o cálculo do desempenho de motores foguete. No presente ponto de controle 2, o fluxo de calor na câmara de combustão de motores-foguete com geometria cilíndrica será calculado utilizando-se das equações newtonianas de temperatura e posteriormente será simulado por meio do programa ANSYS versão R18.1 e R18.2 Academic.

\subsection{Introdução}

Utilizando os conhecimentos anteriormente previstos foram determinados às equações que seriam utilizadas, após as pesquisas. Definimos então a equação da taxa de fluxo de calor, A equação de propagação de calor em cilindros no sentido radial.\\
$$ P_{cond} = \frac{\Delta Q}{\Delta t} = \frac{K.A.\Delta T}{L}$$
$$ Q_{cond} = k_{t}.4.\pi.r_{0}.r_{i}.\frac{T_{S0}-T_{Si}}{r_{0}-r_{i}}$$
